\chapter{Network Security}
\section{Information Gathering}
The first step in attacking a networked environment is always to gather
information on how the network is structured, and which connections, hosts and
segments exist. It is important to know which services exist, as they might
represent possible attack vectors. The next step is identifying vulnerabilities,
and exploiting them. This often includes searching for systems with known
vulnerabilities. What follows is often some measure to ensure permanent control,
such as the installation of a rootkit or backdoor. This process may then be
repeated to gain control to additional systems.

\subsection{DNS}
DNS exposes a lot of information about a network. Brute-forcing reverse DNS
lookups for a targets IP range reveals domains or hostnames, which often reveal
the purpose or service of a system.

\subsection{whois}
\texttt{whois} provides information about an IP range, specifically details
about the organization and administrator responsible. Privacy regulations make
it possible to vastly reduce the information visible nowadays.

\subsection{traceroute}
Network topology can be examined using \texttt{traceroute}. This tool uses ICMP
packets with increasing TTL to find the address of each intermediate router
between the source and a target address. Executing this from multiple hosts to
multiple targets enables mapping the complete network.

\subsection{SMTP}
\subsection{Sniffing}
Tools such as tcpdump or wireshark allow passive recording and decoding of
network traffic. This allows direct retrieval of information transmitted in
plaintext, and a lot of information even with encrypted communication through
metadata.

\subsection{Scanning}
Active scanning of hosts and networks is possible through tools like
\texttt{nmap}. The goal is usually to find out on which TCP or UDP ports there
are applications listening.

\paragraph{TCP connect scanning} simply tries to establish a TCP connection with
the target. The typical packet sequence in a successful case would be
\texttt{SYN}, \texttt{SYN/ACK}, \texttt{ACK}. If no application is listening,
the sequence would be \texttt{SYN}, \texttt{RST/ACK}. This scanning technique is
fast and easy to execute, even for non-superuser users. Disadvantages include
that this only works for TCP, and is easy to detect even on application level.

\paragraph{TCP \texttt{SYN} scanning} is an alternative where no complete TCP
session is ever created. If the port is open and the target responds with
\texttt{SYN/ACK}, the connection is immediately aborted with \texttt{RST}, not
completing the three-way-handshake. This never creates a connection that the
application receives, making it a lot more difficult to detect. This however
also only works for TCP, and additionally requires superuser privileges to
execute.

\section{Attacks}
The TCP/IP family of protocols was not designed with security as an objective.
Various different attacks are possible:

\subsection{Routing Attacks}
Attacks on routing protocols can facilitate redirection of traffic through the
attackers systems, or blackhole the traffic for a denial of service attack.
Routing protocols currently in place are not very security focused, OSPF for
example uses simple authentication using plaintext passwords and MD5 hashes,
BGP4 relies on manual filters for incoming information.

Inside a subnet, techniques such as ARP spoofing are possible, which allow
redirection of traffic to the attackers machine just by responding to ARP
queries for the corresponding IP address (or the gateway). DHCP replies can be
forged in a similar way.

\subsection{DNS Attacks}
DNS attacks are of special importance since basically all internet communication
relies on DNS during initialization of the connection. If the attacker controls
the DNS server, they can redirect traffic intended for a specific domain to a
malicious server. One method of attacking is \emph{DNS cache poisoning}:

DNS makes use of a hierarchical architecture, and various levels of caching to
ease load on the root and authoritative servers. The attacker targets such a
caching server by triggering it to refresh the cache, for example by querying a
targeted domain, and then immediately delivering a malicious response which then
gets stored in the cache. While refreshing a cache entry, the server sends a
request with a specific \texttt{query ID} to the upstream server. For a response
to be accepted, the \texttt{query ID} has to match, but it is not verified in
any other way if the reply indeed originates from an upstream server instead of
a malicious attacker.

The \texttt{query ID} is traditionally simply incremented each request, which
means the attacker could easily guess it by querying the server for a domain, to
which they own the authoritative name server, and assuming that the
\texttt{query ID} for the to-be-poisoned request will be a subsequent number.
The attacker will send multiple replies with different \texttt{query ID}s, with
a high probability that one of them matches. If the reply arrives at the cache
before the legitimate reply, the attack has succeeded.

The challenges to the attacker are so far:
\begin{itemize}
    \item The entry must not be in the cache already
    \item The correct \texttt{query ID} must be guessed
    \item The response must arrive before the legitimate response
\end{itemize}

The most straightforward fix is to implement cryptographically secure random
\texttt{query ID}s.

It is however still possible to take over an entire nameserver/zone: The goal
here is to corrupt the cache entry for the authoritative nameserver in the
caching nameserver. Requests to the victim nameserver can be made for any
non-existing subdomain the target nameserver is responsible for. In the forged
response, a \texttt{NS} record is present containing the domain of the
legitimate authoritative nameserver, but the accompanying \texttt{A} record
points to a server the attacker controls. The ability to use arbitrary
subdomains that will not be cached makes guessing the \texttt{query ID}
feasible, combined with the fact that it is only 16bit long.

Additional mitigations include using the port number of the request as an
additional identifier, but this does only increase the effort in guessing, it
does not solve the underlying problem. A long term solution is DNSSEC
(\cref{sec:dnssec}).


\subsection{Man in the Middle Attacks}

\subsection{Denial of Service Attacks}

\section{Security Mechanisms}

\subsection{DNSSEC}
\label{sec:dnssec}
The basic idea of DNSSEC is that the authoritative servers sign the records
using asymmetric cryptography. The caching resolvers can then verify the
signatures. The required key-hierarchy lends itself to the hierarchical model
which is already in place with DNS. A server signs its records with a \emph{zone
signing key}, which is signed using the servers \emph{key signing key}. A hash
of the KSK is available at the parent zones nameserver, making it possible to
verify the identity of the server up to the root zone.

\subsection{Firewalls}

\subsection{Intrusion Detection}
