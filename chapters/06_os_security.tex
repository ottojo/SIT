\chapter{OS Security}
\textit{Operating system-} and \emph{host-security} has the goals of protecting
stored data, running processes and the operating system itself. This
necessitates some form of authentication, to distinguish between users on a
multi-user system, as well as an access control system which assigns permissions
to users. A useful concept applied here is isolation, which is applied to users
and processes. Raw hardware access is usually restricted, and only allowed to
privileged code within the operating system. Protection against external access
may also be part of host security.

\section{Concepts and Reference Monitors}
A \emph{reference monitor} is an abstract machine which mediates all accesses to
objects by subjects (see \cref{chapter:access_control}). Reference monitors can
be implement on any level of the system. Reference monitors in the systems
hardware include the MMU and privileged execution modes. Kernel level reference
monitors are for example implemented in the file system or capability system.
Applications can also contain reference monitors, which may be the case in web
server applications, or run completely inside another program which implements a
reference monitor such as the Java virtual machine or a database engine.

The \emph{security kernel} is the hardware, firmware and software of a trusted
computing base which implements the reference monitor. It must mediate all
accesses, be protected from modification and be verifiable as correct.

The \emph{trusted computing base (TCB)} is the totality of protection mechanisms
within a computer system. This includes hardware, firmware and software which is
responsible for enforcing a security policy. The enforcement of the security
policy must only depend on the TCB, the rest of the operating system need not to
be trusted.

\section{Virtualization and Mandatory Access Control}
\subsection{Virtualization}
There exist many reasons to use virtualization, but the goal of virtualization
from a security perspective is full isolation of systems and applications. The
possibility to roll back an entire system to a known-good state in case of
compromise also presents an advantage.

Levels of virtualization are distinguished based on the role and position of the
\emph{virtual machine monitor VMM}. In \emph{native virtualization}, the VMM
directly interfaces with the hardware. In \emph{user mode virtualization}, the
VMM only interfaces with the host OS, and not directly with the hardware. A
hybrid approach is \emph{dual-mode virtualization}, where a host OS exists, but
some form of direct access to the hardware by the VMM is possible.

The interaction of the VMM with the \emph{guest OS} provides another way of
categorization: In \emph{full virtualization}, the guest OS runs unmodified, as
on real hardware. This is usually assisted by various hardware extensions such
as \textit{AMD-V} and \textit{Intel VT-x}, special support by the MMU and
passthrough of system busses such as PCI.

\emph{Paravirtualization} refers to a mode of virtualization in which the guest
OS is aware of the virtualization and has some adaptations to the host OS.
Hardware drivers for example can be replaced with components that directly
interface with the VMM.

\subsection{Isolation}
Isolation is the main benefit of virtualization from a security perspective.
Errors in applications can be contained effectively, programs with different
security requirements can be separated, and malware analysis is possible without
effecting the host system. The host system can employ detailed monitoring, and
perform effective intrusion detection from the outside. The isolation of common
dependencies between applications prevents the compromise of multiple
applications by compromise of the common dependency.

As an example, a payment processor may be located in a dedicated VM with a
secure operating system. The non-critical application such as a frontend which
exposes a large attack surface runs in a separate VM and interfaces with the
payment application only through a closely monitored interface.

\subsection{Security Enhanced Linux}
Linux provides the \emph{Linux Security Module (LSM)} interface. Once a user
triggers a security sensitive activity in the kernel through a syscall (for
example: \texttt{read}), the LSM hook is executed, which the LSM module
registers. The LSM module can then for example deny the filesystem access.
\emph{SELinux} is such a module. SELinux supports a variety of mandatory access
control schemes, the main one discussed here is \emph{type enforcement}. The
idea is that both subjects and objects are assigned security labels. A central
security policy determines which subjects can execute which operations on the
objects based on the labels. The acting entity is always a process. Processes
are assigned to \emph{domains}, for which access rules to objects are specified.
SELinux supports two modes of handling processes without a specified domain:
assigning all of those to a default \texttt{unconfined} domain or creating an
individual domain for each process.

Files are tagged with security \emph{labels}. The label my include a user identity,
role, type or domain. The last field implements the type enforcement: for
regular files, this expresses the type of the object, but for executables this
determines the domain of the process once it is executed.

Beyond the simple type enforcement, \emph{roles} can be defined. Users can have
roles, and users can change the current role if they have sufficient privilege.
Specific rights can then be given to specific roles.

Multi-category and multi-level security is also implemented in SELinux,
implementing the Bell-Lapadula model (see \cref{sec:belllapadula}).

\subsection{AppArmor}
\todo[inline]{AppArmor: describe example from video 6b}

\section{Use-Case: iOS Security}
\todo[inline]{iOS Security}
