\chapter{Access Control}
\label{chapter:access_control}
Access control combines authentication and authorization.
\Cref{chapter:authentication} showed how the identity of a subject can be
confirmed, the question is now whether the subject is \emph{authorized} to
access a specific resource. This decision is generally performed by a
\emph{reference monitor}, on the basis of some \emph{access policy}, which may
be set by an administrator or the owner of the object.To differentiate the terms
\textit{subject} and \textit{object}:

A \emph{subject} is the \textit{acting entity}, which intends to carry out an
operation requiring access. Examples include persons, processes or network
nodes. A subject can also be the object of an access operation at another time.

An \emph{object} is the unit that is \textit{being accessed}. Examples include
files and directories, database records, computers, processes, mailboxes or
applications.

The \emph{access operation} is the type of operation of a subject in relation to
an object. Different sets of access operations can be defined. A file system
might for example define \textit{read} and \textit{write} operations (or
\textit{observe} and \textit{alter}). There might be additional operation like
\textit{execute} or \textit{append}, which can be handled independently.

\section{Access Control Matrix}
The access policy can be described using an \textit{access control matrix}. The
matrix contains the set of every allowed operation for every possible pair of
subject and object. This however immediately presents a problem, as this
approach scales badly. For $N$ subjects (users) and $M$ objects (files), a $N
\times M$ matrix would have to be stored.

Two implementations of the access control matrix may be more appropriate
depending on the application:

\subsection{Access Control List}
In an Access Control List (\textit{ACL}), a list of subject-operation pairs is
stored with each object. A file might for example contain a list of all users
that have access to that file, and which exact operations they are allowed to
execute on the file. If a user is not allowed to interact with the file in any
way, the entry may be omitted.

\subsection{Capabilities}
In direct contrast to the ACL, in a capabilities based system the subjects
contain a list of objects and the corresponding operations which the subject is
authorized to execute on the object. In the file system example, each user would
contain a list of files which the user is allowed to access.

\section{Models for Access Control}
More detailed access control models are in use:
\begin{itemize}
    \item[DAC] \emph{Discretionary Access Control} is the approach as explained
          above. Access control decisions are based on a subject and object.
          Subjects define the access rights of an object (file owner for
          example).
    \item[MAC] \emph{Mandatory Access Control} is oriented towards clearance
          levels, as one might expect in a government or military setting. The
          access rights are usually defined system-wide, and are not changed by
          a subject. More information on this in \cref{sec:belllapadula}.
    \item[RBAC] In \emph{Role Based Access Control} access to objects is not
          defined per subject, but per \textit{role}. Roles are given to
          subjects, and may change.
    \item[ABAC] \emph{Attribute Based Access Control} is even more abstract,
          where both subjects and objects have attributes, and access control
          decisions are made by flexible comparison of attributes.
\end{itemize}

Real world systems often implement aspects of multiple of those models, and no
model is the best or definitive answer to access control.

\section{Example: Linux}
\todo[inline]{Linux access control, ACLs}

\section{Multilevel Security, Bell-Lapadula-Model}
\label{sec:belllapadula}
In multilevel security, both objects and subjects are classified. Classes may
for example be \textit{Unclassified}, \textit{Confidential}, \textit{Secret} and
\textit{Top Secret}. Access is only granted if the subject has at least the same
classification as the object. Access is denied if the subject has a lower
classification than the object. The \emph{Bell-Lapadula-Model} is an
implementation of this access control model:

To define this model, consider a set of subjects $S$, a set of objects $O$ and
the set of operations $A = \{\text{execute}, \text{read}, \text{append},
\text{write}\}$. For two security levels $a, b \in SL$, there always exists a
greatest lower bound $l \in SL: (l \leq a, l \leq b \text{ and } l
\text{minimal})$ and a least upper bound $h$.

The system is in a state at all times. A state is a triple $(b, M, f)$ with
\begin{itemize}
    \item $b \subseteq S \times O \times A$ the set of current accesses
    \item $M = (M_{so})_{s\in S, o\in O}$ the current access matrix
    \item $f = (f_s, f_c, f_o)$ with
          \begin{itemize}
              \item $f_s: S \mapsto SL$ the maximum clearance of each subject
              \item $f_c: S \mapsto SL$ the current clearance of each subject
                    (this requires $f_c(s) \leq f_s(s)$ for each subject $s$)
              \item $f_o: O \mapsto SL$ the security classification of each
                    object
          \end{itemize}
\end{itemize}

The current clearance $f_c$ is chosen by each subject, within the limit set by
the maximum clearance $f_s$.

A state is secure, if the following properties are met:
\begin{itemize}
    \item \emph{Simple Security Property}: For all $(s, o, a) \in b$ with
          $a=\text{read}$ or $a=\text{write}$: $f_o(o) \leq f_s(s)$ (Each
          subject which is currently reading or writing has the necessary
          maximum clearance)
    \item \emph{*-Property}:
          \begin{itemize}
              \item For all $(s,o,a)\in b$ with $a=\text{append}$: $f_c(s) \leq
                        f_o(o)$ \\ (No append operation is executed with higher
                        clearance than the object)
              \item For all $(s,o,a)\in b$ with $a=\text{write}$: $f_c(s) =
                        f_o(o)$ \\ (Each write operation is executed with
                        minimal clearance)
              \item For all $(s,o,a)\in b$ with $a=\text{read}$: $f_c(s) \geq
                        f_o(o)$ \\ (Each read operation is executed with a
                        sufficient current clearance)
          \end{itemize}
          This property enforces the correct selection of the current clearance
          $f_c$. Appending happens ``upwards'', writing on the same level, and
          reading ``downwards''.
    \item \emph{Discretionary Security Property}: For all $(s,o,a) \in b$: $A
              \in M_{s,o}$
\end{itemize}

\section{Multicategory Security}
\todo[inline]{multi category security}

\section{POSIX Capabilities}
POSIX capabilities are an implementation of a capability based access control
scheme. An example would be the right to open raw sockets on linux, which is
usually reserved to the root user. This is however necessary for the
\texttt{ping} utility, which should be accessible to every user, even without
setting the \texttt{setuid} bit on the binary. The POSIX capability system now
allows giving the specific right to open raw sockets to the specific binary.
This also adheres to the principle of least privilege a lot better than always
executing \texttt{ping} with full root rights. Capabilities are also not
reserved to files, but can be given to individual processes as well.
