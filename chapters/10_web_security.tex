\chapter{Web Security}
All components of a web application, the browser, the web server and the
underlying infrastructure are all subject to various attacks:

The \emph{connection} is vulnerable to misuse of TCP/IP mechanisms, such as
\textit{SYN flooding}. Eavesdropping and man-in-the-middle attacks are of
concern to any network connection, and as such also for web applications.

\emph{Web servers} or \emph{web applications} are often the target of an attack.
Various techniques such as XSS, injections and authentication bypasses are
explained later in this chapter.

Attacks against the \emph{browser} exploit flaws in the renderer, engine,
through scripting or plugins like java and flash.

\section{Transport Layer Security}
TLS presents a solution to secure TCP/IP connections. Its goals are to provide
authentication, protection, and confidentiality: Servers are authenticated using
X.509 certificates, clients can optionally authenticate the same way. The
transferred data is encrypted. TLS is implemented on top of TCP/IP and provides
an API very similar to normal TCP sockets.

\subsection{Handshake Protocol}
The TLS handshake protocol establishes a TLS session, including authentication,
negotiation of cryptographic primitives and negotiation of a symmetric session
key. One TLS session allows for multiple parallel TLS connections.

\begin{figure}
    \centering
    \begin{tikzpicture}
        \begin{umlseqdiag}
            \umlactor[class=Client]{c}
            \umlobject[class=Server]{s}
            \begin{umlfragment}[type=1, inner xsep=2, name=1]
                \begin{umlcall}[op={client hello}, dt=7]{c}{s}
                \end{umlcall}
                \begin{umlcall}[op={server hello}]{s}{c}
                \end{umlcall}
            \end{umlfragment}
            \umlnote[x=7, y=-2]{1}{Negotiation of security parameters (``ciphersuite'')}


            \begin{umlfragment}[type=2, inner xsep=2, name=2]
                \begin{umlcall}[op=certificate,dt=7]{s}{c}
                \end{umlcall}
                \begin{umlcall}[type=return, op={server key exchange}]{s}{c}
                \end{umlcall}
                \begin{umlcall}[type=return, op={certificate request}]{s}{c}
                \end{umlcall}
                \begin{umlcall}[op={server done}]{s}{c}
                \end{umlcall}
            \end{umlfragment}
            \umlnote[x=7, y=-4]{2}{Server authentication towards client}
            \umlnote[x=7, y=-6]{2}{Optional: key exchange and request for client certificate}

            \begin{umlfragment}[type=3, inner xsep=2, name=3]
                \begin{umlcall}[type=return, op={certificate},dt=7]{c}{s}
                \end{umlcall}
                \begin{umlcall}[op={client key exchange}]{c}{s}
                \end{umlcall}
                \begin{umlcall}[type=return, op={certificate verify}]{c}{s}
                \end{umlcall}
            \end{umlfragment}
            \umlnote[x=7, y=-8]{3}{Client key exchange}
            \umlnote[x=7, y=-10]{3}{Optional: certificate-based client authentication}

            \begin{umlfragment}[type=4, inner xsep=2, name=4]
                \begin{umlcall}[op={change cipher spec},dt=7]{c}{s}
                \end{umlcall}
                \begin{umlcall}[op={finished}]{c}{s}
                \end{umlcall}
                \begin{umlcall}[op={change cipher spec}]{s}{c}
                \end{umlcall}
                \begin{umlcall}[op={finished}]{s}{c}
                \end{umlcall}
            \end{umlfragment}
            \umlnote[x=7, y=-12]{4}{Switch to encryption}
        \end{umlseqdiag}
    \end{tikzpicture}
    \caption{TLS Handshake Protocol}
    \label{fig:tls_handshake}
\end{figure}

\begin{figure}
    \centering
    \begin{tikzpicture}
        \begin{umlseqdiag}
            \umlactor[class=Client]{c}
            \umlobject[class=Server]{s}
                \begin{umlcall}[op={client hello}, dt=7]{c}{s}
                    \begin{umlcall}[op={client random},dt=5]{c}{s}
                    \end{umlcall}
                    \begin{umlcall}[op={suggested cipher suites}]{c}{s}
                    \end{umlcall}
                    \begin{umlcall}[op={suggested compression}]{c}{s}
                    \end{umlcall}
                    \begin{umlcall}[op={session ID: 0x00}]{c}{s}
                    \end{umlcall}
                \end{umlcall}
                \begin{umlcall}[op={server hello}]{s}{c}
                    \begin{umlcall}[op={server random},dt=5]{s}{c}
                    \end{umlcall}
                    \begin{umlcall}[op={use cipher suites}]{s}{c}
                    \end{umlcall}
                    \begin{umlcall}[op={session ID}]{s}{c}
                    \end{umlcall}
                \end{umlcall}
        \end{umlseqdiag}
    \end{tikzpicture}
    \caption{TLS Handshake Protocol: Hello messages (phase 1)}
    \label{fig:tls_hello}
\end{figure}

\Cref{fig:tls_handshake} shows an overview of the TLS handshake: In \emph{phase
1}, the client offers a list of supported cipher suites to the server. The
server selects a cipher suite, and establishes a session ID. Random values for
later key creation are also exchanged. \Cref{fig:tls_hello} shows phase 1 in
detail.

In \emph{phase 2}, the server sends its certificate, and optionally requests a
client certificate for authentication of the client.

In \emph{phase 3}, the client sends its certificate, if requested. The client
has verified the server using the certificate, and can now use the severs public
key for establishing a symmetric session key. The client generates a
\texttt{PreMasterSecret}, from which client and server can derive the symmetric
key using the random values exchanged in phase 1. The \texttt{PreMasterSecret}
is RSA-encrypted using the servers public key and then sent to the server.

In \emph{phase 4}, the symmetric session key is established. The integrity of
the handshake is verified by exchanging hashes over the derived key and the
previous messages. All following communication happens encrypted.

\subsection{Record Protocol}
Once the handshake is complete, the handshake protocol takes over and handles
transfer of data. The record protocol is completely separate to the handshake
protocol, making it theoretically possible to execute the handshake completely
offline. The record protocol handles, in order: Fragmentation, Compression,
Message Authentication Code, Encryption. Receiving works the same way in the
opposite order. MAC and encryption use separate keys for both directions.

\section{Injection Attacks}
In injection attacks, some specially crafted input to the web application leads
to unwanted effects.

\subsection{Cross Site Scripting}
Many web applications accept some kind of textual input, which then gets shown
to a different user on the website, such as a forum- or social media post. If no
special care is taken, this means that anyone can place arbitrary HTML,
including scripts, on the website in the targets browser. This is referred to as
\textit{Cross Site Scripting} or \emph{XSS}. A typical goal of such an attack is
to steal cookies, which might allow the attacker to take over the login session
of the target. Exfiltration of the cookie might happen via the query parameters
of an image that gets loaded from the attackers server via the injected script.
Protection against those kinds of attack include validating the input, and
removing possibly dangerous characters, or making sure that user generated
content is only interpreted as text by the browser, and not executed. Other
measures include instructing the targets browser to not make any request to
external servers, which hinders exfiltration.

\subsection{SQL Injection}
User inputs such as login credentials are often used as parts of SQL queries. If
the input contains valid SQL syntax, and is just appended to a query, it might
change the query in a way such that it always returns a certain value to bypass
authentication, or execute arbitrary statements with the privilege of the web
server application. Even escaping the input data might not solve the issue, as
second order injections might circumvent that. Again, it is important to ensure
input data is only handled as text and no situation exists where it is
potentially executed as code.
